\chapter{Conclusions and Future Work}\label{ch:Conclusion}
\section{Summary and Conclusions}
In this thesis, a collective emitting model based on Green function theory and FDTD method has been presented, and verified for few-dipole cases, and applied to many-dipole cases. We treated the dipoles as two-level fermion systems. Under the single excitation and the equally excited mixed state conditions, the optical properties of ensemble-coupling optical cavity system have been studied. Through this study, a nanophotonic Matlab toolbox was composed and can be used in a wide range of studies.
This thesis has investigated the spectra of a large number of emitters coupled to an optical cavity, and studied the modifications of the cavity spectrum through the collective emission. It has shown that, as the pumping rates in the ME model increases, the number of excited dipoles is increased as a result. The scattering and emitting processes among the dipoles can modify the optical property of the cavities dramatically. It can narrow or broaden the cavity spectrum, shift the cavity resonance, and enhance the coupling strength between a target dipole and the cavity, all of which are dependent on the central position and width of the dipole ensemble, the coupling strengths and the decay rates of individual dipoles. While coupled to the cavity, the background dipoles are also shifted, weakened or enhanced by the cavity. All of these results are consistent with reported experimental observations.

Furthermore, the interactions between the background dipoles and the cavity are through the target dipoles, if any, which have a relatively large coupling strength and are close to the cavity resonance. The position and coupling strength of the target dipole can affect the coupling of the background dipoles to the cavity in return.

This research gives a better understanding of optical cavity theory. The conventional models usually correlate the number of physical emitters, such as QDs, to the number of dipoles. However, this research finds that the number of dipoles which creates the excitons also depends on the pumping environment, which is hidden in the phenomenological parameters of pumping rates in the previous models. Moreover, the conventional cavity models usually also assume that the mode of an optical cavity is mainly predetermined by the bare cavity. However, this research shows that the existence of the luminescent ensemble can also modify the optical properties of the cavity and hence the coupling between a target emitter and the cavity, in the context of scattering and collective emission. Notice that when we are referring to the cavity, we only employ the information of bare cavity spectrum and field strength. The discussions above may also suitable for waveguide cases with a given input pulse mode.

Notice that the GF model this thesis employed is under weak excitation approximation, where the linear optical assumption is held. A nonlinear GF model has already been presented in Ref.~\onlinecite{Tureci2008} to explain the modes competition effect for random lasers. Similar conclusions should be held for nonlinear cases; that is, one should consider the modification effects of excited excitons and pumping effects when analyzing cavity emissions.

The results presented above may provide a fundamental perspective of optical control of cavity and waveguide, as well as individual emitters, based on collective emission in optical cavities. Applications such as an increase in the robustness of a quantum network using random modes and an atom ensemble have been recently proposed~\cite{sapienza2010cavity,wiersma2010random}.

We believe that, all of the conclusions implied by the scientific research above have answered the questions I have listed in the introduction to this thesis.


%It concludes that:
%a) Individual emitters can broaden the cavity spectrum dramatically as the coupled dipoles number arises;

%b) Background emitters coupled to an optical cavity can act as a pumping drive, which is linearly coupled to the exciton spectrum and nonlinearly coupled to the cavity spectrum;

%c) When all emitters have an identical resonance, photon emitters ensemble can also narrow down the cavity spectrum as the ensemble number grows up. However, the trend of spectral width as a function of ensemble number is nonlinear and fluctuated.

%d) The strength of coupled ensemble effect depends on how close the ensemble resonances to cavity resonance, compared with their spectrum width. And the emitters which is much close to cavity resonance can change the cavity's optical property much more, compared with those far away from the cavity resonance. In this case, emitters in different resonance ranges can be approximately treated as ensemble "bins" or one single equivalent dipole.

%e) When a photon emitter very closes to the cavity resonance, we may not view it as a single dipole, or coupled-bin model fails for on-resonance dipoles ensemble.

\section{Future Work}
As we have discussed in the introduction, to fully understand the excitons-cavity interactions and better design functional cavity devices, one may need to consider the collective emitting and light-scattering effects which can greatly modify the optical properties of a cavity system. From the conclusions of this study, one may develop a better way of designing optical cavity devices as follows: first, calculate the bare cavity property using numerical tools such as FDTD software; second, statistically measure the background exciton distribution to be located in the device, either by considering the material growth condition and historical records, or by actually making a trial device sample merely with background emitters, and measure the spectrum; next, calculate the background exciton modification to the cavity system and optimize the resonance location of the target emitters. In this way, an on-demand device can be made.

Theoretically, there are many aspects that one can consider to depict the dipoles-cavity interactions. One is that by implementing the multi-level model of emitters into the GF method, it can adapt to many general cases of emitters. The MEs and rate equations may be helpful for this trial. Another aspect is that nonlinear GF can be considered for high-excitation cases, following the study of Tureci in Ref.~\onlinecite{Tureci2008}, for instance.

Since the Dyson equations used in this thesis are a general form which can be used to study some many-body physics processes, phonon-cavity interactions~\cite{Wilson-Rae2002,Brandes2005} may also be able to take into account in our model under the same algorithm.

As has been claimed in Sections~\ref{section:GFT} and~~\ref{section:GN}, there are relationships between the GFs and the electric susceptibility, which fully determines the optical properties of a medium. We only studied some aspects of the optical properties of optical cavities in this thesis. One should be able to study the other aspects of optical cavities, such as the spontaneous emission coefficients, the gain and lasing characteristics, and the dispersion of a cavity~\cite{Chow1999,Du2004}. Applications such as slow light devices~\cite{Vlasov2005,Baba2008} using collective emitters can be discussed in the context of GFs.

The strong-coupling criterion of exciton-cavity interaction can be studied under the schema presented in this thesis. As noted, the earlier studies are mainly based on identical atoms or other emitters, but the cases when background weakly coupled excitons were present may have not been well studied.

The cross-coupling effect between different polarization directions is another interesting topic for further study, as we have pointed out in the beginning of Chapter~\ref{ch:cavity}. A careful study on this topic may answer the question of whether the random orientation of background excitons affects the polarization of emitted photon from a cavity coupling with a strong target dipole. To solve this problem, the $xy$ component of GFT should be calculated for a given cavity, such as a micropillar cavity. Alternatively, the correlation function between $x$ and $y$ components of the GFT or polarized spectrum is needed. The collective emission of elliptical QDs and cavities~\cite{Austing1999,Tokura2001,Jinxi2001,Accatino1997}, for example, can be better studied with the consideration of the cross-coupling effect.

Coherent excited ensemble effects can be also discussed within a similar framework. This discussion can be extended to some practical quantum information applications, such as entanglement generation and quantum control, the research on which has been speeded up in the past decade.

Finally, pumping stabilization is another interesting topic to study: that is, how to keep the optical properties of a cavity stable regardless of pump power. Collective emission may be a way to address this topic.

%I believe, we are all on the way to exploring, behaving, and changing, individually and collectively. There is always more to do, beyond paperwork and lab-based research. Sometimes we need to stop, and unambitiously listen to what the nature tells, what is the destiny of human beings and of oneself.  Knowledge is dormant until we use it in our lives. What kind of world we are growing into is determined by all of us...

%\section{Conclusion}
%This study explains the experimental observation on cavity spectrum broadening and narrowing effects with coupled photon emitters ensemble. For the cavity spectrum narrowing effect with a large number of coupled identical emitters, it may have potential applications for quantum information processing with photonic implements, such as NV centers coupled cavity system, since a narrower spectrum means a better coherence. The nanophotonic toolbox developed along this study has potential usage for researches in nearby topics. 