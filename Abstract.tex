This thesis presents a theoretical study of the collective effects of a large number of photon emitters coupled to optical cavities. The ensemble effects are accounted for by considering both the light emitting and scattering by the photon emitters. It suggests that, to correctly estimate the emitters ensemble coupled cavity mode, it is necessary to consider the existence of the excited excitons ensemble and optical pumps. This thesis shows that optical pumps can excite more excitons and scattering channels as pumping power increases. The change in exciton population can lead to comprehensive spectral behaviors by changing the cavity spectral shapes, bandwidth and resonance positions, through the inhomogeneous broadening and frequencies repulsion effects of collective emissions. The existence of the exciton ensemble can also enhance optical coupling effects between target excitons and the cavity mode. The target exciton, which has a relatively large coupling strength and is close to the cavity peak, can affect the properties of the background dipoles and their coupling to the cavity. All these collective effects are sensitive to the number, the resonances distribution, and the optical properties of the background excitons in the frequency domain and the property of the target exciton, if any. This study provides a perspective on the control of the optical properties of cavities and individual excitons through collective excitation.
